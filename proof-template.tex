\documentclass[12pt]{article}

\usepackage{fancyhdr} % For the customized heading
\usepackage[english]{babel}
\usepackage[utf8x]{inputenc}
\usepackage[most]{tcolorbox}
\usepackage{multicol}
\usepackage{adjustbox}
\usepackage{amsmath,amsthm,amsfonts,amssymb} % For math symbols
\usepackage{extramarks}
\usepackage{graphicx} % For adding pictures
\usepackage{booktabs}
\usepackage{tabu}
\usepackage[T1]{fontenc}
\usepackage{enumitem}
\usepackage[parfill]{parskip}
\usepackage{changepage}
\usepackage{scrextend}
\usepackage{gauss}% http://ctan.org/pkg/gauss
\usepackage{hyperref}
\usepackage{mathtools}
\usepackage{calc}
\usepackage{scrextend}

%% Packages for APA bibliographies 
\usepackage{apacite} % for the references page
\usepackage{url} % for hyperlinks in references

\usepackage{empheq}


\newtcolorbox[auto counter]{lin-proof}[1][]
{colframe=blue!30,
	colback=white,
	sharp corners,
	title=\emph{Proof},
	enhanced,
	coltitle=black,
	fonttitle=\bfseries,
	attach boxed title to top left={yshift*=-\tcboxedtitleheight/2, xshift=3mm},
	boxed title style={sharp corners, colback=blue!30},
	#1 
}

\newtcolorbox[auto counter]{solution}[1][]
{colframe=blue!20,
	colback=white,
	sharp corners,
	title=Solution,
	enhanced,
	fontupper=\linespread{1.0}\selectfont,
	coltitle=black,
	fonttitle=\bfseries,
	attach boxed title to top left={yshift*=-\tcboxedtitleheight/2, xshift=3mm},
	boxed title style={sharp corners, colback=blue!20},
	#1
}

\newtcbox{\boxedeq}[1][]{%
    nobeforeafter, math upper, tcbox raise base,
    enhanced, colframe=black,
    colback=white, boxrule=0.5pt,
    #1}
\newenvironment{amatrix}[1]{
	\left[\begin{array}{@{}*{#1}{c}|c@{}}
}{
		\end{array}\right]
}
%% BASIC DOCUMENT SETTINGS %%

\graphicspath{ {./images/} } 

\topmargin=-0.45in
\evensidemargin=0in
\oddsidemargin=0in
\textwidth=6.5in
\textheight=9.0in
\headsep=0.25in

\linespread{1.1}

\pagestyle{fancy}
\lhead{\hmwkAuthorName}
\chead{}
\rhead{\hmwkTitle}
\lfoot{\lastxmark}
\cfoot{\thepage}

\newcommand{\hmwkTitle}{\textbf{MATH 208 HW 4.1: Subspaces}}
\newcommand{\hmwkDueDate}{January 17, 2019}
\newcommand{\hmwkClass}{\textbf{MATH 238 - Differential Equations}}
\newcommand{\hmwkClassTime}{Section A}
\newcommand{\hmwkClassInstructor}{Timothy Trammel}
\newcommand{\hmwkAuthorName}{\textbf{Isabel Giang}}
\newcommand{\continueNextPage}{\begin{flushright}\textit{Continued on the next page...}
\end{flushright} \pagebreak}
\renewcommand\headrulewidth{0.4pt}
\renewcommand\footrulewidth{0.4pt}

\setlength\parindent{0pt}


%% TITLE PAGE SETTINGS %%

\title{
    \vspace{2in}
    \textmd{\textbf{\hmwkTitle}}\\
    \large\vspace{0.1in}{\hmwkClass}\\
    \large\vspace{0.1in}{\textbf{Date: }\hmwkDueDate}\\
    \vspace{0.1in}{\textbf{Professor: }\hmwkClassInstructor}
    \vspace{1in}
}

\author{\hmwkAuthorName}
\date{}

\renewcommand{\vec}[1]{\mathbf{#1}}
\renewcommand\qedsymbol{$\blacksquare$}
\newcommand{\lintrans}{T: \mathbf{R^{m}} \rightarrow \mathbf{R^{n}}}

\newenvironment{question}[1]{
\textbf{#1}.
}


\NewEnviron{elaboration}{
\par
\begin{tikzpicture}
\node[rectangle,minimum width=0.9\textwidth] (m) {\begin{minipage}{0.85\textwidth}\BODY\end{minipage}};
\draw[dashed] (m.south west) rectangle (m.north east);
\end{tikzpicture}
}

\begin{document}

\begin{elaboration}

\textbf{Definition of a Subspace}:
A subset $S$ of $\vec{R^n}$ is a subspace if $S$ satisfies the following three conditions:

\begin{enumerate}[label=(\roman{enumi})]
\item $S$ contains $\vec{0}$, the zero vector.
\item If $\vec{u}, \vec{v}$ are in $S$, then $\vec{u + v}$ is in $S$. This is known as \emph{closure under addition}.
\item If $\vec{u}$ is in $S$, then $r\vec{u}$ is in $S$ for every real number $r$. This is known as \emph{closure under scalar multiplication.}
\end{enumerate}


We can determine if this subset is a subspace by checking if it satisfies all three of these conditions. 

We can also use \textbf{Theorem 4.2}, which states: Let $S$ = span($\{\vec{u_1}, \vec{u_2},..., \vec{u_m} \}$) be a subset of $R^n$. Then $S$ is a subspace of $R^n$.

\end{elaboration}

\pagebreak
\section*{Proofs}

\textbf{63.} Prove that if $\vec{b} \neq 0$, then the set of solutions to $A\vec{x} = \vec{b}$ is not a subspace.

\begin{lin-proof}
We will prove this by assuming the contrapositive.

Assume that the set of solutions to $A\vec{x} = \vec{b}$ is in fact a subspace. If this is the case, then the set of solutions must contain the zero vector $\vec{0}$. 

This also means that the zero vector $\vec{0}$ must satisfy the matrix equation $A\vec{x} = \vec{b}$, such that $A\vec{0} = \vec{x}$.

 We know this isn't true, because $A\vec{0} = \vec{0}$, which means that b must be $\vec{0}$, but it is given that $b \neq \vec{0}$. 
 
Thus, since the only way for the set of solutions to be a subspace is for it to contain the zero vector is if $b = \vec{0}$, which is not true, the set of solutions cannot be a subspace. 


 \begin{flushright}$\blacksquare$ \end{flushright}
\end{lin-proof}

\textbf{69.}  Let $A$ be a matrix and $T(\vec{x}) = A\vec{x}$ a linear transformation. Show that $ker(T) = \{\vec{0}\}$ if and only if the columns of $A$ are linearly independent.

\begin{lin-proof}

The transformation $T(\vec{x}) = A\vec{x}$ is linear, so we know that the following conditions must be satisfied:~\\

\begin{elaboration}
\textbf{Definition of a Linear Transformation}
\begin{enumerate}[label=\alph*)]
\item $T(\vec{u} + \vec{v}) = T(\vec{u}) + T(\vec{v})$

\item $T(c\vec{u}) = cT(\vec{u})$
\end{enumerate}

\end{elaboration}

Recall that the kernel of $T$, ker$(T)$ is the set of vectors $\vec{x}$ such that $T(\vec{x}) = \vec{0}$.

Also recall that in order for a set of vectors to be linearly independent, the only solution to the vector equation
$$\vec{x_1} + \vec{x_2} + \dots + \vec{x_n} = \vec{0}$$ must be the trivial solution. 

Since ker$(T)$ is the set of vectors that satisfies T$(\vec{x}) = \vec{0}$ and the matrix equation $A\vec{x} = 0$ corresponds with this matrix, the only vector in ker$(T)$ must be the trivial solution.

So ker$(T)$ must be equal to $\vec{0}$ - in other words, the trivial solution - if and only if the columns of $A$ are linearly independent.
 \begin{flushright}$\blacksquare$ \end{flushright}
\end{lin-proof}
\pagebreak
\textbf{70.} If $T$ is a linear transformation, show that $\vec{0}$ is always in ker($T$).

\begin{lin-proof}

The transformation $T$ is linear, so we know that the following conditions must be satisfied:~\\

\begin{elaboration}
\textbf{Definition of a Linear Transformation}
\begin{enumerate}[label=\alph*)]
\item $T(\vec{u} + \vec{v}) = T(\vec{u}) + T(\vec{v})$

\item $T(c\vec{u}) = cT(\vec{u})$
\end{enumerate}

\end{elaboration}

Recall that the kernel of $T$, ker$(T)$ is the set of vectors $\vec{x}$ such that $T(\vec{x}) = \vec{0}$.


Since $T(c\vec{u}) = cT(\vec{u})$ by the definition of a linear transformation, we know that $T(0\vec{u}) = 0T(\vec{u})$ This must equal the zero vector $\vec{0}$.

Since $T(0) = \vec{0}$, the zero vector is part of the set of vectors where $T(\vec{x}) = \vec{0}$, so the zero vector $\vec{0}$ must be in ker$(T)$.

 \begin{flushright}$\blacksquare$ \end{flushright}
\end{lin-proof}

\textbf{71.} Prove that if $u$ and $v$ are in a subspace $S$, then so is $\vec{u - v}$.

\begin{lin-proof}
If $\vec{u}$ and $\vec{v}$ are in the subspace $S$, this means that all linear combinations of $\vec{u}$ and $\vec{v}$ are also in the subspace, since the definition of a subspace requires that the closure properties - closure under addition and closure under scalar multiplication - be satisfied.

Particularly, since $S$ is closed under addition, $\vec{u} + \vec{-v} = \vec{u + v}$ must be in $S$.

The subtraction of vectors can be understood as the addition of negative scalar multiples, which is satisfied by the closure properties.
 \begin{flushright}$\blacksquare$ \end{flushright}
\end{lin-proof}

\pagebreak
\textbf{72.} Prove Theorem 4.6: If $T$ is a linear transformation, then $T$ is one-to-one if and only if ker($T$) = $\{\vec{0}\}$.

\begin{lin-proof}

\begin{elaboration}
\textbf{Theorem 4.6}

Given a linear transformation $T: \vec{R^{m}} \rightarrow \vec{R^{n}}$, $T$ must be one-to-one if and only if ker$(T)$ = $\lbrace \vec{0} \rbrace$.

\end{elaboration} ~\\

\underline{Prove that $T$ being one-to-one implies that ker$(T)$ = $\lbrace \vec{0} \rbrace$.}

\begin{addmargin}[1em]{2em}

Assume that $T$ is one-to-one. This means that there is at most one solution to $T(\vec{x})$. We know this because of Theorem 3.5. ~\\

\begin{elaboration}
\textbf{Theorem 3.5}

Let $T$ be a linear transformation. $T$ is one-to-one if and only if $T(\vec{x}) = \vec{0}$ has only the trivial solution $\vec{x} = \vec{0}$.

\end{elaboration}

Since T is a linear transformation, we know that $T(\vec{0}) = 0$ since $T(c\vec{0}) = cT(\vec{0}) = \vec{0}$.
Thus, ker($T$) = $\lbrace \vec{0} \rbrace$~\\

\end{addmargin}


\underline{Prove that  implies that ker$(T)$ = $\lbrace \vec{0} \rbrace$ implies that $T$ is one-to-one.}

\begin{addmargin}[1em]{2em}

Assume that ker$(T)$ only contains the trivial solution in its set of vectors.

If $T(\vec{u}) = T(\vec{v})$, then $T(u) - T(v) = 0$.
This implies that $T(\vec{u} - \vec{v})$ = $\vec{0}$. We know this is true since T is a linear transformation, and  $T(\vec{u} + \vec{v}) = T(\vec{u}) + T(\vec{v})$.


Since $T(\vec{x}) = \vec{0})$ only has the trivial solution, if ker$(T)$ = $\lbrace \vec{0} \rbrace$, it follows that $\vec{u - v} = \vec{0}$ and $\vec{u} = \vec{v}$. Thus, $T$ is one-to-one.


\end{addmargin}

 \begin{flushright}$\blacksquare$ \end{flushright}
\end{lin-proof}



\end{document}

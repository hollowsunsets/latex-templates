\documentclass[10pt,landscape]{article}
\usepackage[utf8]{inputenc}
\usepackage{multicol}
\usepackage{calc}
\usepackage{enumerate}
\usepackage{ifthen}
\usepackage[landscape]{geometry}
\usepackage{graphicx}
\usepackage{tabu}
\usepackage{amsmath,amsthm,amsfonts,amssymb}
\DeclareMathOperator{\arcsec}{arcsec}
\DeclareMathOperator{\arccot}{arccot}
\DeclareMathOperator{\arccsc}{arccsc}
\DeclareMathOperator{\Sf}{Sf}
\DeclareMathOperator{\gr}{gr}
\usepackage{color,graphicx,overpic}
\usepackage{hyperref}
\usepackage{mathtools}
\usepackage{tikz}
\usepackage{relsize}
\newcommand*\circled[1]{\tikz[baseline=(char.base)]{
            \node[shape=circle,draw,inner sep=1pt] (char) {#1};}}
\newcommand{\RNum}[1]{\uppercase\expandafter{\romannumeral #1\relax}}
\newcommand{\Lagr}{\mathcal{L}}
\graphicspath{ {./images/} } 

\pdfinfo{
  /Title (MATH 153 Quiz 6 Cheat Sheet)
  /Creator (Isabel Giang)
  /Producer (Isabel Giang)
  /Author (Isabel Giang)
  /Subject (Parametric Equations)
  /Keywords (pdflatex, latex,pdftex,tex)}

\ifthenelse{\lengthtest { \paperwidth = 11in}}
    { \geometry{top=.5in,left=.5in,right=.5in,bottom=.5in} }
    {\ifthenelse{ \lengthtest{ \paperwidth = 297mm}}
        {\geometry{top=1cm,left=1cm,right=1cm,bottom=1cm} }
        {\geometry{top=1cm,left=1cm,right=1cm,bottom=1cm} }
    }

\pagestyle{empty}

\makeatletter
\renewcommand{\section}{\@startsection{section}{1}{0mm}%
                                {-1ex plus -.5ex minus -.2ex}%
                                {0.5ex plus .2ex}%x
                                {\normalfont\large\bfseries}}
\renewcommand{\subsection}{\@startsection{subsection}{2}{0mm}%
                                {-1explus -.5ex minus -.2ex}%
                                {0.5ex plus .2ex}%
                                {\normalfont\normalsize\bfseries}}
\renewcommand{\subsubsection}{\@startsection{subsubsection}{3}{0mm}%
                                {-1ex plus -.5ex minus -.2ex}%
                                {1ex plus .2ex}%
                                {\normalfont\small\bfseries}}
\makeatother

\def\BibTeX{{\rm B\kern-.05em{\sc i\kern-.025em b}\kern-.08em
    T\kern-.1667em\lower.7ex\hbox{E}\kern-.125emX}}

\setcounter{secnumdepth}{0}

\setlength{\parindent}{0pt}
\setlength{\parskip}{0pt plus 0.5ex}

\newtheorem{example}[section]{Example}
% -----------------------------------------------------------------------

\begin{document}
\raggedright
\footnotesize
\begin{multicols*}{3}

\setlength{\premulticols}{1pt}
\setlength{\postmulticols}{1pt}
\setlength{\multicolsep}{1pt}
\setlength{\columnsep}{2pt}

\begin{center}
     \textbf{\large{\underline{MATH 153 Ch. 10 \& 12  Formula Sheet}}} \\
\end{center}

\section*{Conic Sections}

For unshifted standard forms of conic sections, let $h = 0, k = 0$.
\begin{center}	
	{\tabulinesep=0.5mm
	\textbf{Parabolas}
	\begin{tabu} to \columnwidth {||c | c | c||}
	\hline
		Graph & \includegraphics[scale=0.25]{parabola1} & 
				\includegraphics[scale=0.25]{parabola2} \\
		\hline
		Equation & $(x - h)^2 = 4p(y - k)$ & $(y - k)^2 = 4p(x - h)$ \\ 
		{} & $y = (x - h)^2 + k$ & $x = (y - k)^2 + h$\\
		\hline
		Directrix & $y = k - p$ & $x = h - p$ \\
		\hline
		Foci  & $(h, k + p)$ & $(h + p, k)$ \\
		\hline
		Vertices & $(h, k)$ & $(h, k)$ \\
	\hline
	\end{tabu}}
\end{center}
\begin{center}	
	{\tabulinesep=0.5mm
	\textbf{Ellipses} $\mathbf{\frac{(x - h)^2}{a^2} + \frac{(y - k)^2}{b^2} = 1}$
	\begin{tabu} to \columnwidth {||c | c | c||}
	\hline
		Graph & \includegraphics[scale=0.15]{ellipse1} & 
				\includegraphics[scale=0.15]{ellipse2} \\
		{} & {$a \geq b > 0$} & {$b \geq a > 0 $} \\
		\hline
		Center & $(h, k)$ & $(h, k)$ \\
		\hline
		Foci  & $(h \pm c, k)$ & $(h, k \pm c)$ \\
		{} & $ \scriptstyle (c^2 = a^2 - b^2)$ & $ \scriptstyle (c^2 = b^2 - a^2)$  \\
		\hline
		Vertices & $(h \pm a, k)$ & $(h, k \pm b)$ \\
		\hline
	\hline
	\end{tabu}}
\end{center}
\begin{center}	
	{\tabulinesep=0.5mm
	\textbf{Hyperbolas}
	\begin{tabu} to \columnwidth {||c | c | c||}
	\hline
		Graph & \includegraphics[scale=0.15]{hyperbola1} & 
				\includegraphics[scale=0.15]{hyperbola2} \\
		\hline
		Equation & $\frac{(x - h)^2}{a^2} - \frac{(y - k)^2}{b^2} = 1$ &
				   $\frac{(y - k)^2}{b^2} - \frac{(x - h)^2}{a^2} = 1$\\
		\hline
		Center & $(h, k)$ & $(h, k)$ \\
		\hline
		Foci  & $(h \pm c, k)$ & $(h, k \pm c)$ \\
		$ \scriptstyle (c^2 = a^2 + b^2)$ & {} & {}  \\
		\hline
		Vertices & $(h \pm a, k)$ & $(h, k \pm b)$ \\
		\hline
		Asympt. &  $y = k \pm \frac{b}{a}(x - h)$ & $y = k \pm \frac{b}{a}(x - h)$ \\
	\hline
	\end{tabu}}
\end{center}
\vfill
\columnbreak

\section{Parametric Equations}
General Form of Parametric Equations
$$ x = f(t), \quad y = g(t), \quad a \leq t  \leq b$$
Parametric Equations of an Ellipse
$$ x = a \cos(\omega)t, \quad y = b \sin(\omega)t$$
Traces an ellipse exactly once in a counter-clockwise direction starting at the point $(a, 0)$ in the range $0 \leq t \leq 2\pi$
\subsection{Parametric Derivatives}
$$ \frac{dy}{dx} = \mathlarger{\frac{\frac{dy}{dt}}{\frac{dx}{dt}}}, \text{where } \frac{dx}{dt} \neq 0$$
$$ \frac{d^2y}{dx^2} = \frac{d}{dx}\left(\frac{dy}{dx}\right) = \frac{\frac{d}{dt}\left(\frac{dy}{dx}\right)}{\frac{dx}{dt}}$$
\subsection{Parametric Tangents}
Tangent Line: $\mathlarger{y = F(a) + m(x - a) \text{ where } \frac{d}{dx}\Bigr|_{\substack{t=a}} = F'(a)}$
Horizontal Tangents: $\mathlarger{\frac{dy}{dt} = 0,\text{where } \frac{dx}{dt} \neq 0} $

Vertical Tangents: $\mathlarger{\frac{dx}{dt} = 0, \text{where } \frac{dy}{dt} \neq 0} $
\subsection{Parametric Areas Under Curves}
$$ A = \int^{\beta}_{\alpha} g(t) f'(t) \text{ dt}, \quad a = f(\alpha), b = f(\beta)$$

$$ A = \int^{\alpha}_{\beta} g(t) f'(t) \text{ dt}, \quad b = f(\alpha), a = f(\beta) $$
where $\alpha$ and $\beta$ are values of $t$, $a$ and $b$ are values of $x$, $y$.
\subsection{Parametric Arc Length}

$$ L = \int^{\beta}_{\alpha} \sqrt{\left(\frac{dx}{dt} \right)^2 + \left(\frac{dy}{dt} \right)^2} \text{ dt}$$ 

The calculated arc length may not be proportional to the number of times the parametric curve has been traced. 
\subsection{Parametric Surface Areas}
$$S = \int 2\pi{y} \text{ ds} \quad \text{ rotation about x-axis} $$
$$S = \int 2\pi{x} \text{ ds} \quad \text{ rotation about y-axis} $$
$$\text{ds} = \sqrt{\left(\frac{dx}{dt}\right)^2 + \left(\frac{dy}{dt} \right)^2} \text{ dt } \quad \text{ if } x = f(t), y = g(t), \alpha \leq t \leq \beta $$
\vfill
\columnbreak

\section{Polar Coordinates}

Alt. Representations of Polar Coordinates
$$(r, \theta + 2\pi n)$$
$$(-r, \theta + (2\pi + 1)n)$$

Polar to Cartesian Formula
$$x = r \cos(\theta) $$
$$y = r \sin(\theta) $$

Cartesian to Polar Formula
$$r^2 = x^2 + y^2; \quad r = \sqrt{x + y} $$ 
$$\theta = \tan^{-1}\left( \frac{y}{x} \right) $$

\subsection{Basic Polar Graphs}


\begin{center}
\textbf{Symmetry Tests}\\
{\tabulinesep=0.5mm
	\begin{tabu} to \columnwidth {||c | c ||}
			\hline
			Symmetric about the polar axis or x-axis & $(r, -\theta)$ \\
			Symmetric about the pole/origin & $(-r, \theta)$ \\
			Symmetric about $\theta = \pi/2$ or y-axis & $(-r, -\theta)$ \\
		\hline
	\end{tabu}}
\end{center}

\begin{center}	
	\textbf{Lines} \\
	{\tabulinesep=0.5mm
	\begin{tabu} to \columnwidth {||c | c ||}
		\hline
			Polar & Cartesian \\
			\hline
			$\theta = \beta$ & $y = (\tan \beta) x$ \\
			$r\cos \theta = a$ & $x = a$ \\
			$r\sin \theta = b$ & $y = b$ \\
		\hline
	\end{tabu}}
\end{center}

\begin{center}	
	\textbf{Circles} \\
	{\tabulinesep=0.5mm
	\begin{tabu} to \columnwidth {||c | c ||}
		\hline
			Polar & Cartesian \\
			\hline
			$r = a$ & $x^2 + y^2 = a^2$ \\
			$r = 2a\cos\theta$ & $(x - a)^2 + y^2 = (\left|a \right|)$ \\
			$r = 2b\sin\theta$ & $x^2 + (y - b)^2 = (\left|b \right|)$ \\
		\hline
	\end{tabu}}
\end{center}

\subsection{Polar Tangents}
$$ \frac{dy}{dx} = \frac{\frac{dr}{d\theta} \sin \theta + r \cos \theta}{\frac{dr}{d\theta} \cos\theta - r\sin\theta} $$
\subsection{Polar Areas Under Curves}
$$A = \int^{\beta}_{\alpha} \frac{1}{2}r^2 d\theta$$
\subsection{Polar Arc Length}
$$L = \int ds = \int \sqrt{r^2 + \left(\frac{dr}{d\theta}\right)^2} d\theta $$ 


\subsection{Polar Conic Sections}

Horizontal Conic: $\mathlarger{r = \frac{ed}{1 \pm e\cos\theta}, \quad x \pm d}$ 

~\\

Vertical Conic:
$ \mathlarger{r = \frac{ed}{1 \pm e\sin\theta}, \quad y \pm d}$


~\\

The conic is an ellipse if $e < 1$, a parabola if $e = 1$ or a hyperbola if $e > 1$.

\section*{3D Coordinate Space}
Distance Between Two Points in 3D
$$d(P_1, P_2) = \sqrt{(x_2 - x_2)^2 + (y_2 - y_1)^2 + (z_2 - z_1)^2} $$

Eq. of Sphere with center $(h, k, l)$ and radius $r$:
$$ (x - h)^2 + (y - k)^2 + (z - l)^2 = r^2 $$

\section*{Vectors}
Magnitude of a Vector: $\quad |\vec{a}\| = \sqrt{a_1^2 + a_2^2 + \dots + a_n^2}$~\\

Direction of a Vector: $ \theta = \arctan\left(\frac{\vec{a}_x}{\vec{a}_y} \right) $~\\

Unit Vector: $\quad \mathlarger{\vec{u} = \frac{1}{|\vec{a}|}\vec{a}}$~\\

~\\

Net Angle Between Two Vectors 
$$\cos \theta = \frac{\vec{a} \cdot \vec{b}}{|\vec{a}| |\vec{b}|} $$
Direction Cosines
$$\frac{1}{|a|}\vec{a} = \langle \cos \alpha, \cos \beta, \cos \gamma \rangle $$
Scalar projection of $\vec{b}$ onto $\vec{a}$:
$$\text{comp}_{\vec{a}}\vec{b} = \frac{\vec{a} \cdot \vec{b}}{|\vec{a}|}$$
Vector projection of $\vec{b}$ onto $\vec{a}$:
$$\text{proj}_{\vec{a}}\vec{b} = \frac{\vec{a} \cdot \vec{b}}{|\vec{a}^2|}\vec{a}$$
\subsection*{Dot Product}
Given two vectors $\vec{a} = \langle a_1, a_2, \dots, a_n \rangle$ and $\vec{b} = \langle b_1, b_2, \dots, b_n \rangle$, the dot product $\vec{a} \cdot \vec{b}$ is:
$$\vec{a} \cdot \vec{b} =  a_1b_1 + a_2b_2 + \dots + a_nb_n  $$
$$ \vec{a} \cdot \vec{b} = |a||b|\cos\theta $$
\subsection*{Cross Product}
Given two vectors $\vec{a} = \langle a_1, a_2, a_3 \rangle$ and $\vec{b} = \langle b_1, b_2, b_3 \rangle$ in $\mathbf{R^{3}}$, the cross product $\vec{a} \times \vec{b}$ is:

$$\vec{a} \times \vec{b} = \langle a_2b_3 - a_3b_2, a_3b_1 - a_1b_3, a_1b_2 - a_2b_1 \rangle $$
$$|\vec{a} \times \vec{b}| = |a||b|\sin\theta $$

\vfill
\columnbreak
\section{Other Formulas}
\subsection{Exponent Properties}
$$a^ma^n = a^{n + m}$$
$$(a^{n})^m = a^{nm}$$
$$\frac{a^n}{a^m} = a^{n - m}$$
$$(ab)^n = a^{n}b^{n}$$
$$\left(\frac{a^{n}}{b^{m}} \right) = \frac{a^{nk}}{b^{mk}}$$
$$\frac{1}{a^{-n}} = a^{n}$$
\subsection{Logarithm Properties}
$$\log_b(M \cdot N) = \log_b(M) + \log_b(N)$$
$$\log_b\left(\frac{M}{N}\right) = \log_b(M)- \log_b(N)$$
$$\log_b(M^{k}) = k\cdot \log_b(M) $$
\subsection{Trignometric Identities}
$$\sin^2 x + \cos^2 x = 1$$
$$1 + \tan^2 x = \frac{1}{\cos^2 x}$$
$$\cos^2 x = \frac{1 + \cos(2x)}{2}$$
$$\sin^2 x = \frac{1 - \cos(2x)}{2}$$
$$\sin(2x) = 2\sin(x)\cos(x) $$
$$\cos(2x) = \cos^2(x) - \sin^2(x)$$
$$\qquad = 2\cos^2(x) - 1$$
$$\qquad = 1 - 2\sin^2(x)$$

\subsection{Trigonometry Reference}

\begin{center}	
	\textbf{Common Values of Sine and Cosine} \\
	~\\
	{\tabulinesep=0.5mm
	\begin{tabu} to \columnwidth {||c | c | c ||}
		\hline
			$\theta$ & Cosine & Sine \\
			\hline
			0        & 1            & 0\\
			$\pi/6$  & $\sqrt{3}/2$ & $1/2$\\
			$\pi/4$  & $\sqrt{2}/2$ & $\sqrt{2}/2$ \\
			$\pi/3$  & $1/2$        & $\sqrt{3}/2$\\
			$\pi/2$  & $0$          & $1$\\
			$\pi$    & $-1$         & $0$ \\
			$3\pi/2$ & $0$          & $-1$ \\
		\hline
	\end{tabu}}
\end{center}


%\clearpage
%\rule{0.3\linewidth}{0.25pt}
%\scriptsize
%\bibliographystyle{abstract}
%\bibliography{refFile}





\end{multicols*}
\end{document}

\documentclass[12pt]{article}

\usepackage{fancyhdr} % For the customized heading
\usepackage[english]{babel}
\usepackage[utf8x]{inputenc}
\usepackage[most]{tcolorbox}
\usepackage{multicol}
\usepackage{adjustbox}
\usepackage{amsmath,amsthm,amsfonts,amssymb} % For math symbols
\usepackage{extramarks}
\usepackage{graphicx} % For adding pictures
\usepackage{booktabs}
\usepackage{tabu}
\usepackage[T1]{fontenc}
\usepackage{enumitem}
\usepackage[parfill]{parskip}
\usepackage{changepage}
\usepackage{scrextend}
\usepackage{gauss}% http://ctan.org/pkg/gauss
\usepackage{hyperref}
\usepackage{mathtools}
\usepackage{calc}
\usepackage{scrextend}
\usepackage{systeme}
\usepackage{empheq}


%% Packages for APA bibliographies 
\usepackage{apacite} % for the references page
\usepackage{url} % for hyperlinks in references


%% BASIC DOCUMENT SETTINGS %%

\graphicspath{ {./images/} } 

\topmargin=-0.45in
\evensidemargin=0in
\oddsidemargin=0in
\textwidth=6.5in
\textheight=9.0in
\headsep=0.25in

\linespread{1.1}

\pagestyle{fancy}
\lhead{\hmwkAuthorName}
\chead{}
\rhead{\hmwkTitle}
\lfoot{\lastxmark}
\cfoot{\thepage}

\newcommand{\hmwkTitle}{\textbf{MATH 208: Take-Home Assignment}}
\newcommand{\hmwkDueDate}{January 17, 2019}
\newcommand{\hmwkClass}{\textbf{MATH 208 - Linear Algebra}}
\newcommand{\hmwkClassTime}{Section A}
\newcommand{\hmwkClassInstructor}{Dana Updegrove}
\newcommand{\hmwkAuthorName}{\textbf{Isabel Giang}}
\newcommand{\continueNextPage}{\begin{flushright}\textit{Continued on the next page...}
\end{flushright} \pagebreak}
\renewcommand\headrulewidth{0.4pt}
\renewcommand\footrulewidth{0.4pt}

\setlength\parindent{0pt}


%% TITLE PAGE SETTINGS %%

\title{
    \vspace{2in}
    \textmd{\textbf{\hmwkTitle}}\\
    \large\vspace{0.1in}{\hmwkClass}\\
    \large\vspace{0.1in}{\textbf{Date: }\hmwkDueDate}\\
    \vspace{0.1in}{\textbf{Professor: }\hmwkClassInstructor}
    \vspace{1in}
}

\author{\hmwkAuthorName}
\date{}

%% TODO:
                   
%% MATH STRUCTURES AND SHORTCUTS       %% 

%% Solution environment
\newtcolorbox[auto counter]{solution}[1][]
{colframe=blue!20,
	colback=white,
	sharp corners,
	title=Solution,
	enhanced,
	fontupper=\linespread{1.0}\selectfont,
	coltitle=black,
	fonttitle=\bfseries,
	attach boxed title to top left={yshift*=-\tcboxedtitleheight/2, xshift=3mm},
	boxed title style={sharp corners, colback=blue!20},
	#1
}


%% Proof environment
\newtcolorbox[auto counter]{lin-proof}[2][]
{colframe=blue!20,
	colback=white,
	sharp corners,
	title=Proof: #2,
	enhanced,
	coltitle=black,
	fonttitle=\bfseries,
	attach boxed title to top left={yshift*=-\tcboxedtitleheight/2, xshift=3mm},
	boxed title style={sharp corners, colback=blue!20},
	#1
}

%% Boxed answer environment
\newtcbox{\boxedeq}[1][]{%
    nobeforeafter, math upper, tcbox raise base,
    enhanced, colframe=black,
    colback=white, boxrule=0.5pt,
    #1}

%% Augmented matrix environment
\newenvironment{amatrix}[1]{
	\left[\begin{array}{@{}*{#1}{c}|c@{}}
}{
		\end{array}\right]
}

%% Vector notation
\renewcommand{\vec}[1]{\mathbf{#1}}
%% "Tombstone" QED for proofs
\renewcommand\qedsymbol{$\blacksquare$}
%% Linear transformation notation
\newcommand{\lintrans}{T: \mathbf{R^{m}} \rightarrow \mathbf{R^{n}}}

\begin{document}

\textbf{Problem 2, page 43}

The volume of traffic for a collection of intersections is show in the figure below. Find all possible values for $x_1, x_2, x_3$ and $x_4$. What is the minimum volume of traffic from $C$ to $D$?

\includegraphics[scale=0.35]{prob21-4}

\begin{solution}
The number of cars entering each intersection must equal the number of cars leaving. There are four intersections, so  we can get the following system of equations:
\begin{align*}
70 + x_2 &= 85 + 40 + x_1 \label{A} \\
40 + 20 + 25 + x_3 &= 30 + x_2 \\
100 + x_1 &= 70 + x_4 \\
30 + x_4 &= x_3 + 60
\end{align*}
We can rewrite these equations into the standard form of a system of linear equations:

\begin{alignat*}{4}
-x_1  & {}+{} & x_2  & {}{} & {}{}  & {}{} & {}={} & 55 \\
 {}{} & {}{} &  -x_2 & {}+{} &  x_3 & {}{} & {}={} &  -55 \\
  x_1 & {}{} & {}{} & {}{} & {}{} & {}{} & -x_4 {}={} & -30 \\
 {}{} & {}{} & {}{} & {}{} & x_3 & {}+{} & x_4 {}={} & 30
\end{alignat*}

which can be converted into an augmented matrix and converted to echelon form:
$$\begin{bmatrix} -1 & 1 & 0 & 0 & 55 \\ 
				   0 & -1 & 1 & 0 & -55\\ 
				   1 & 0 & 0 & -1 & -30 \\
				   0 & 0 & -1 & 1 & 30 \\
				   \end{bmatrix} \Rightarrow 
\begin{bmatrix} -1 & 1 & 0 & 0 & 55 \\ 
				   0 & -1 & 1 & 0 & -55\\ 
				   0 & 0 & 1 & -1 & -30 \\
				   0 & 0 & 0 & 0 & 0 \\
				   \end{bmatrix}$$
From this, we can get the following set of equations:
\begin{alignat*}{4}
-x_1  & {}+{} & x_2  & {}{} & {}{}  & {}{} & {}={} & 55 \\
 {}{} & {}{} &  -x_2 & {}+{} &  x_3 & {}{} & {}={} &  -55 \\
 {}{} & {}{} & {}{} & {}{} & x_3 & {}+{} & x_4 {}={} & 30
\end{alignat*}

Using back substitution and letting $x_4 = s_4$ as a free parameter, the set of solutions is: $$x_1 = s_4 - 30,000, \quad x_2 = s_4 + 25,000, \quad x_3 = s_4 - 30,000, \quad x_4 = s_4$$

Thus there are infinitely many possible distributions of cars for the four intersections.

It is not possible for a volume of cars to be negative, so $x_4 \geq 30,000.$ ~\\

The minimum volume of cars from $C$ to $D$ is \textbf{30,000 cars.}

\end{solution}
\pagebreak

\textbf{Problem 4, page 43}

The volume of traffic for a collection of intersections shown in the figure below. Find all possible values for $x_1, x_2, x_3, x_4, x_5$ and $x_6$. 

\includegraphics[scale=0.35]{prob41-4}

\begin{solution}
There are six intersections, so we will have six equations to represent the volume of traffic:
\begin{align*}
x_2 + 50 &= 20 + 40 + x_2 \\
x_3 + x_4 + 20 &= x_2 + 45 \\
45 + 60 &= x_4 + x_5 + 35 \\
x_1 + 60 &= 80 \\
80 &= 70 + x_3 \\ 
x_5 + 70 &= x_6
\end{align*}
We can rewrite these equations into the standard form of a system of linear equations:
$$\sysdelim..\systeme{
  -x_1  + x_2                         = 10,
         -x_2 + x_3 + x_4             = 25,
                      x_4 + x_5       = 70,
   x_1                                = 20,
                x_3                   = 10,
                            x_5 - x_6 = -70
}$$
which can be converted into an augmented matrix and converted to echelon form:
$$\begin{bmatrix} -1 & 1 & 0 & 0 & 0 & 0 & 55 \\ 
				   0 & -1 & 1 & 1 & 0 & 0 & -55\\ 
				   0 & 0 & 0 & 1 & 1 & 0 & 70 \\
				   1 & 0 & 0 & 0 & 0 & 0 & 20 \\
				   0 & 0 & 1 & 0 & 0 & 0 & 10 \\
				   0 & 0 & 0 & 0 & 1 & -1 & -70
				   \end{bmatrix} \Rightarrow 
\begin{bmatrix}   -1 & 1 & 0 & 0 & 0 & 0 & 10 \\ 
				   0 & -1 & 1 & 1 & 0 & 0 & 25\\ 
				   0 & 0 & 1 & 1 & 0 & 0 & 55 \\
				   0 & 0 & 0 & 1 & 1 & 0 & 70 \\
				   0 & 0 & 0 & 0 & 1 & 0 & 25 \\
				   0 & 0 & 0 & 0 & 0 & -1 & -95
				   \end{bmatrix} $$
From this, we can get the following set of equations:

$$\sysdelim..\systeme{
  -x_1  + x_2                         = 10,
         -x_2 + x_3 + x_4             = 25,
                x_3 + x_4             = 55,
                      x_4 + x_5       = 70,
                            x_5       = 25,
                                 -x_6 = -95
}$$


Using back substitution, the set of solutions is: $$\mathbf{x_1 = 20, \quad x_2 = 30, \quad x_3 = 10, \quad x_4 = 45, \quad x_5 = 25, \quad x_6 = 95}$$
\end{solution}


Use Example 5 as a guide to find the subspace of values that balances the given chemical equation.

\textbf{Problem 74, page 160}

Methane burns in oxygen to form carbon dioxide and steam.

$$x_1CH_4 + x_2O_2 \rightarrow x_3CO_2 + x_4H_2O$$

\begin{solution}
In order to balance the chemical equation, we need to find values of $x_1$, $x_2$, $x_3$, $x_4$, such that the number of atoms for each element in the equation is the same for both sides.

We can create the linear system:
$$\sysdelim..\systeme{
  -x_1 -        x_3                   = 0,
  4x_1                        -2x_4   = 0,
         2x_2 - 2x_3           -x_4   = 0
}$$

This can be converted into the following augmented matrix, which can be reduced into echelon form:

$$\begin{bmatrix} 1 & 0 & -1 & 0 & 0 \\
				   4 & 0 & 0 & -2 & 0 \\
				   1 & 2 & -2 & -1 & 0 \end{bmatrix} \Rightarrow 
\begin{bmatrix} 1 & 0 & 0 & -\frac{1}{2} & 0 \\
				 0 & 1 & 0 & -1 & 0 \\
				 0 & 0 & 1 & -\frac{1}{2} & 0 \end{bmatrix} $$
				 
From this we can get the following set of solutions:
$$ x_1 = \frac{1}{2}s_4, \quad x_2 = s_4, \quad x_3 = \frac{1}{2}s_4,\quad  x_4 = s_4$$ where $s_4$ is a free parameter.

We can use this to get a set of values that will satisfy the chemical equation. This is the subspace of values that balances the equation:

$$\text{span}\left\lbrace \begin{bmatrix} \frac{1}{2} \\ 1 \\ \frac{1}{2} \\ 1 \end{bmatrix} \right\rbrace $$


\end{solution}

\pagebreak
\textbf{Problem 76, page 160}

Ethyl alcohol reacts with oxygen to form vinegar and water.

$$x_1C_2H_5OH + x_2O_2 \rightarrow x_3HC_2H_3O_2 + x_4H_2O$$

\begin{solution}
In order to balance the chemical equation, we need to find values of $x_1$, $x_2$, $x_3$, $x_4$, such that the number of atoms for each element in the equation is the same for both sides.

We can create the linear system:
$$\sysdelim..\systeme{
  2x_1 -          2x_3                   = 0,
  5x_1 -          4x_3           -2x_4   = 0,
   x_1 +   2x_2  -2x_3            -x_4   = 0,
   x_1 -           x_3                   = 0
}$$

This can be converted into the following augmented matrix, which can be reduced into echelon form:

$$\begin{bmatrix} 2 & 0 & -2 & 0 & 0 \\
				   5 & -4 & 0 & 0 & 0 \\
				   1 & 2 & -2 & -1 & 0 \\
				   1 & 0 & -1 & 0 & 0 \end{bmatrix} \Rightarrow 
\begin{bmatrix}  1 & 0 & 0 & -\frac{2}{3} & 0 \\
				  0 & 1 & 0 & -\frac{5}{6} & 0 \\
				  0  & 0 & 1 & -\frac{2}{3} & 0 \\
				   0 & 0 & 0 & 0 & 0 \end{bmatrix} $$
				 
From this we can get the following set of solutions:
$$ x_1 = \frac{2}{3}s_4, \quad x_2 = \frac{5}{6}s_4, \quad x_3 = \frac{2}{3}s_4,\quad  x_4 = s_4$$ where $s_4$ is a free parameter.

We can use this to get a set of values that will satisfy the chemical equation. This is the subspace of values that balances the equation:

$$\text{span}\left\lbrace \begin{bmatrix} \frac{2}{3} \\ \\ \frac{5}{6} \\ \\ \frac{2}{3} \\ \\ 1 \end{bmatrix} \right\rbrace $$



\end{solution}

\pagebreak
\textbf{Problem 28, page 44}

Find the values of the coefficients $a$, $b$, and $c$ so the given conditions for the function $f$ and its derivatives are met.

$$f(x) = ae^{x} + be^{2x} + ce^{-3x}; f(0) = 2, f'(0) = 1,\text{ and } f''(0) = 19.$$


\begin{solution}

Compute the first and second derivatives of the function $f(x)$:

$$ f'(x) = ae^{x} + 2be^{2x} - 2ce^{-3x}$$
$$  f''(x) = ae^{x} + 4be^{2x} + 9ce^{-3x} $$

Plug in $x = 0$ to cancel out the $e^{x}$ terms since $e^{0} = 1$:
\begin{align*}
f(0) &= a + b + c \\ 
f'(0) &= a + 2b - 3c \\
f''(0) &= a + 4b + 9c 
\end{align*}

Set these equations equal to the given initial conditions to form a system of equations:
\begin{align*}
a + b + c &= 2 \\ 
a + 2b - 3c &= 1 \\
a + 4b + 9c &= 19
\end{align*}

This can be converted to the following augmented matrix, which can be reduced to echelon form:

$$ \begin{bmatrix} 1 & 1 & 1 & 2\\
					1 & 2 & -3 & 1\\
					1 & 4 & 9 & 19 \end{bmatrix} \Rightarrow 
\begin{bmatrix} 1 & 0 & 0 & -2\\
					0 & 1 & 0 & 3 \\
					0 & 0 & 1 & 1 \end{bmatrix}$$
					
From this we can obtain values for the coefficents of the equation:
$$a = -2, \quad b = 3, \quad c = 1 $$
\end{solution}
\end{document}